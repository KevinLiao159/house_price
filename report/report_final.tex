%%%%%%%%%%%%%%%%%%%%%%%%%%%%%%%%%%%%%%%%%%%%%%%%%%%%%%%%%%%%%%%%%%%%%%%%%%%%%%%%
%2345678901234567890123456789012345678901234567890123456789012345678901234567890
%        1         2         3         4         5         6         7         8



\documentclass[letterpaper, 10 pt, conference]{ieeeconf}  % Comment this line out
                                                          % if you need a4paper
%\documentclass[a4paper, 10pt, conference]{ieeeconf}      % Use this line for a4
                                                          % paper

\IEEEoverridecommandlockouts                              % This command is only
                                                          % needed if you want to
                                                          % use the \thanks command
\overrideIEEEmargins
% See the \addtolength command later in the file to balance the column lengths
% on the last page of the document



% The following packages can be found on http:\\www.ctan.org
%\usepackage{graphics} % for pdf, bitmapped graphics files
%\usepackage{epsfig} % for postscript graphics files
%\usepackage{mathptmx} % assumes new font selection scheme installed
%\usepackage{times} % assumes new font selection scheme installed
%\usepackage{amsmath} % assumes amsmath package installed
%\usepackage{amssymb}  % assumes amsmath package installed

\title{\LARGE \bf
Predictive modeling for house price prediction
}

%\author{ \parbox{3 in}{\centering Huibert Kwakernaak*
%         \thanks{*Use the $\backslash$thanks command to put information here}\\
%         Faculty of Electrical Engineering, Mathematics and Computer Science\\
%         University of Twente\\
%         7500 AE Enschede, The Netherlands\\
%         {\tt\small h.kwakernaak@autsubmit.com}}
%         \hspace*{ 0.5 in}
%         \parbox{3 in}{ \centering Pradeep Misra**
%         \thanks{**The footnote marks may be inserted manually}\\
%        Department of Electrical Engineering \\
%         Wright State University\\
%         Dayton, OH 45435, USA\\
%         {\tt\small pmisra@cs.wright.edu}}
%}
\author{Minsu Kim, Lingjie Qiao, Kevin Liao, Cheng Peng \\
University of California at Berkeley} %

\usepackage{Sweave}
\begin{document}
\Sconcordance{concordance:report_final.tex:report_final.Rnw:%
1 243 1 50 0 5 1 12 0 34 1}



\maketitle
\thispagestyle{empty}
\pagestyle{empty}


%%%%%%%%%%%%%%%%%%%%%%%%%%%%%%%%%%%%%%%%%%%%%%%%%%%%%%%%%%%%%%%%%%%%%%%%%%%%%%%%
\begin{abstract}

This paper summarizes the background, problem, methodology and results of our analysis on predicting house price in Ames, Iowa. To make full use of statistical models and machine learning techniques we have learned from Stats 159 Reproducible and Collaborative Statistical Data Science, we participate in Kaggle Competition, \textbf{"House Price: Advanced Regression Techniques"}. We use the Ames Housing dataset compiled by Dean De Cock for use in data science education. While the competition only emphasizes the accuracy of predicted values, this paper elaborates thorough explanations on exploratory analysis, feature engineering and statistical modeling. Furthermore, we put our effort on reproducibility of this analysis, so that the reader can reproduce the exact same result from our code.

\end{abstract}


%%%%%%%%%%%%%%%%%%%%%%%%%%%%%%%%%%%%%%%%%%%%%%%%%%%%%%%%%%%%%%%%%%%%%%%%%%%%%%%%
\section{INTRODUCTION}

The House Price project thoroughly explores the predictive modeling process and advanced regression techniques. From previous study, in order to understand the relationship of one dependent variable with several independent variables, we fit a multiple linear regession with Ordinary Least Squares. However, since OLS may have high variance and include irrelevant variables, Predictive Modeling Process can improve the results in terms of Prediction Accuracy and Model Interpretability.

The competition sets the background of the project: Ask a home buyer to describe their dream house, and they probably won't begin with the height of the basement ceiling or the proximity to an east-west railroad. But this playground competition's dataset proves that much more influences price negotiations than the number of bedrooms or a white-picket fence. 

With 79 explanatory variables describing (almost) every aspect of residential homes in Ames, Iowa, this competition requires participants to predict the final price of each home. Our team therefore follows the idea of model prediction and tries to use different techniques in order to most accurately predict the final sales price of each house.

\section{DATA DESCRIPTION}
The datasets are obtained from the Kaggle Competition website. We have access to four files: 
\begin{itemize}
\item \textit{data description}, which provides the official definition for fields.
\item \textit{train.csv}, which provides 1459 real observations that can be used for model construction.
\item \textit{test.csv}, which is used to fit the predictive model and create submission entry for the final sales price of 1460 observations 
\item \textit{sample submission}, which gives an example of how the fitted values should be submitted.
\end{itemize}

The train dataset has in total 80 variables, 79 predictors and 1 target variable called \textit{SalePrice}. We observe both categorial predictors, such as \textit{FireplaceQu}, \textit{GarageCond} and \textit{MasVnrType} as well as numerical predictors, such as \textit{PoolArea}, \textit{EnclosedPorch} and \textit{YrSold}. Since we can potentially create a lot of different new variables, our goal is to understand the relationship between \textit{SalePrice} and these predictors with statistical fitting procedures that minimizes Mean Square Error.

\section{EXPLORATORY DATA ANALYSIS}
to be filled

\section{METHODOLOGY}

The goal of this analysis is to accurately predict the final price of each home. Therefore, we frame this problem as a regression problem, and decide to use the L2 loss function [10] which is often used in regression problem. Taking this objective into account, we preprocess original dataset so that regression models can work well. Furthermore, we extract more features by involving feature engineering. Finally, we fit two shrinkage models and two ensemble models. The details are explained in the following subsections.

\subsection{Evaluation and objective loss function}

We specifically use root mean squared logarithmic error [5], which makes more sense in our problem setting because errors in predicting expensive houses and cheap houses should affect the result equally. The following is the formula of RMSLE.

$$
\epsilon = \sqrt{\frac{1}{n} \sum_{i=1}^{n} (log(p_{i} + 1) - log(a_{i} + 1))^2}
$$

Since L2 loss function minimizes the squared differences between the estimated and existing target values [10], L2 error will be much larger in the case of outlier compared to L1 and therefore L2 loss function is highly sensitive to outliers in the dataset. So, in the preprocessing step, we eliminate outliers to remedy this issue.

\subsection{Preprocess}
According to the exploratory data analysis, we find a lot of NA values in most of predictors. Since regression models cannot handle missing data, we need to either remove or impute data using appropriate methods [7]. The data description provided by client [6] indicates that some of the missing values are actually none value. In that case, we replace NA value with factor variable named None. However, there are some numerical predictors with missing values in an unsystematical manner. In that case, we impute them with mean values of predictors.

As a next step, we apply log transformations to area related predictors such as GrLivArea and LotArea as well as target variable. The log transformation has an effect to remedy skewness of data [8] by making original distributions of predictors to more normally distributed. Consequently, it helps regression model to work better.

Images will be filled

After the log transformation, we still notice few outliers and eliminate lowest and highest 0.1\% data points. Furthermore, majority of predictors are categorical under which regression models cannot be used directly. Therefore, we apply one-hot encoding [9] to convert categorical values to numerical ones. It consequently expands features from 79 to approximately 500. The Table 1 below summarizes this procedure.

Table 1: Data Preprocess and Variable Transformation

\subsection{Data preparation}
Before fitting the model, we first split dataset into train and test. We could have a separate validation set. However,  R library caret has a built-in cross validation as a generic interface. So it automatically takes care of cross validation. We use train data to train and tune our models using 5-kold cross validation, and later compare RMSLE using hold-out test data [13].

\subsection{Featuring engineering}

Considering the complexity of the problem as well as the number of observations and predictors, we assume that the success of this analysis is largely dependent on informative, feature engineered predictors that can reveal the subtle relationship to our target variable. Given the small size of dataset with 1460 observations, we conclude that feature learning, which is a set of techniques that learn a feature: a transformation of raw data input to a representation that can be effectively exploited in machine learning tasks [1], is not a feasible option because feature learning often involves very complicated models with multiple layers, which tends to cause an overfitting issue when dataset is small [3].

With this observation, we therefore focus more on manual feature engineering [4]. This process is a important stepping-stone in that it helps reveal significant predictors that are previously not represented well in original dataset. By explicitly designing what the input x's should be, our predictive models can solve a problem easily. 

\subsection{Model description and hyper-parameter tuning}
While training each model, we need to find optimal paramters for each model. In order to effectively select hyper-parameters, we use 5-fold cross-validation. For lasso and ridge, lambda is the tuning parameter. It determines how much we will penalize models for high weights on predictors. If lambda is high, it penalizes models more and ends up generating sparse models. This kind of models is called shrinkage method because they shrinken weights or even remove predictors by penalizing models. These models are especially good options when there are many predictors. By penalziing or removing unnecessary predictors, they provide more interpretable results. Thus, they are often utilized in genomic and pharametical analysis.

\subsection{Modeling}
We utilize both shrinkage regression models and ensemble models. Practitioners often favor ensemble models [11] due to their conveniences. Ensemble models such as Random Forest [12], which uses the averaged result from the randomly grown decision trees such as CART [13], often work well with unscaled, missing data and are used for both classification and regression problems. Also, since our dataset contains a hug number of predictors, shrinkage methods such Lasso and Ridge regressions [13], which penalize predictors by shrinking their weights, can be highly effective. Furthermore, we utilize a dimension reduction technique called PCA[13] in order to compress information into lower dimension. The results of modeling will be further explained in the following section in detail.

\subsection{Model comparison}
As mentioned on in Evaluation section, we use RMSE to compare models and select the best model. Although RMSE does not provide an absolute means of model accuracy, it provides a relative measure to compare models. Thus, we finalize our model with the lowest RMSE.
In summary, the following is the the procedure for each model.

\begin{enumerate}
\item Split the data into train and test, 80\% and 20\% repectively.
\item Train a model using train data with 5-fold cross-valdation.
\item Pick the optimal hyper-parameters.
\item Predict balance using the model with the optimal parameters.
\item Calculate Mean Square Error.
\item Record both Mean Square Error and coefficients.
\end{enumerate}

\section{ANALYSIS}
Due to the high dimensionality of dataset, we first experiment to decide whether to either project predictors into lower dimensional bases or select predictors using shrinkage methods. First, we try Principal component analysis which is a statistical procedure that uses an orthogonal transformation to convert a set of observations of possibly correlated variables into a set of values of linearly uncorrelated variables called principal components [15]. Since our one-hot encoded dataset has a lot of predictors that have near zero variances, we first need to remove near zero variances. This is because PCA considers variability of data and compress the predictors with high variances into first principal components.   

After removing near zero variance 519 to 128. This indicates that a great number of predictors have apporiximately identical across observations. Figure $model_pca_scree_plot.png$ shows a PCA scree plot for each principal component. It indicates that first few principal components effectively explain the variances of data, and overall graph exponentially decays. Cumulative Proportion in $Table model_pca$ shows that first 10 pcs and 61 pcs explain approximately 44\% and 90\% of variance in dataset respectively. It means that if we decide to reduce dimensionality at the expense of a bit little of prediction accuracy, we could use 61 principal components which are approximately 1/8 of total predictors. 

Furthermore, we try to fit both lasso and regression and tune lambda values using 5-fold cross validation and then examine coefficients. Figure $model_lasso_lambda.png$, $model_ridge_lambda.png$ shows MSE for corresponding lambda values for both models. Figure $model_lasso_lambda.png$ shows that MSE decreases as log(lambda) increase up to log lambda equals -5 and  -1.15 respectively. Based on these results, we decide optimal lambda for both model. We also examine coefficients of lasso regression. Figure $model_lasso_number_of_coefficients_left.png$ indicates that most of weights for corresponding predictors become zeros, which effectively eliminates a great portion of predictors. After elimination, only 78 predictors are left. We plot top 10 coefficients for both lasso and ridge to investigate how much each coefficient contribution to the model. Figure $model_lasso_coefficients_top_10.png$,

$model_ridge_coefficients_top_10.png$ show that top 10 coefficients from both models. It is noteworthy that GrivArea has the highest coefficient in Lasso and YearBuilt2010 has the highest coefficient in Ridge. Both make sense because the area and built year are important aspects when it comes to making a decision about purchasing house. Both models indicate that GrLiArea, NeighborhoodCrawfor and NeighborhoodStoneBr belong to top 10 highest coefficients and are highly associated with house price. Furthermore, we try to figure out feature importance using Gradient boosting machine. Feature importance is automatically measured as GBM grows its tree while minimizing entropy. Figure $model_gbm_predictor importance.png$ shows that overall quality has the highest feature importance 

\section{RESULTS}

\section{CONCLUSIONS}

A conclusion section is not required. Although a conclusion may review the main points of the paper, do not replicate the abstract as the conclusion. A conclusion might elaborate on the importance of the work or suggest applications and extensions. 

\addtolength{\textheight}{-12cm}   % This command serves to balance the column lengths
                                  % on the last page of the document manually. It shortens
                                  % the textheight of the last page by a suitable amount.
                                  % This command does not take effect until the next page
                                  % so it should come on the page before the last. Make
                                  % sure that you do not shorten the textheight too much.

%%%%%%%%%%%%%%%%%%%%%%%%%%%%%%%%%%%%%%%%%%%%%%%%%%%%%%%%%%%%%%%%%%%%%%%%%%%%%%%%



%%%%%%%%%%%%%%%%%%%%%%%%%%%%%%%%%%%%%%%%%%%%%%%%%%%%%%%%%%%%%%%%%%%%%%%%%%%%%%%%



%%%%%%%%%%%%%%%%%%%%%%%%%%%%%%%%%%%%%%%%%%%%%%%%%%%%%%%%%%%%%%%%%%%%%%%%%%%%%%%%
\section*{APPENDIX}

Appendixes should appear before the acknowledgment.

\section*{ACKNOWLEDGMENT}

The preferred spelling of the word ÒacknowledgmentÓ in America is without an ÒeÓ after the ÒgÓ. Avoid the stilted expression, ÒOne of us (R. B. G.) thanks . . .Ó  Instead, try ÒR. B. G. thanksÓ. Put sponsor acknowledgments in the unnumbered footnote on the first page.



%%%%%%%%%%%%%%%%%%%%%%%%%%%%%%%%%%%%%%%%%%%%%%%%%%%%%%%%%%%%%%%%%%%%%%%%%%%%%%%%

References are important to the reader; therefore, each citation must be complete and correct. If at all possible, references should be commonly available publications.



\begin{thebibliography}{99}

\bibitem{c1} G. O. Young, ÒSynthetic structure of industrial plastics (Book style with paper title and editor),Ó 	in Plastics, 2nd ed. vol. 3, J. Peters, Ed.  New York: McGraw-Hill, 1964, pp. 15Ð64.
\bibitem{c2} W.-K. Chen, Linear Networks and Systems (Book style).	Belmont, CA: Wadsworth, 1993, pp. 123Ð135.
\bibitem{c3} H. Poor, An Introduction to Signal Detection and Estimation.   New York: Springer-Verlag, 1985, ch. 4.
\bibitem{c4} B. Smith, ÒAn approach to graphs of linear forms (Unpublished work style),Ó unpublished.
\bibitem{c5} E. H. Miller, ÒA note on reflector arrays (Periodical styleÑAccepted for publication),Ó IEEE Trans. Antennas Propagat., to be publised.
\bibitem{c6} J. Wang, ÒFundamentals of erbium-doped fiber amplifiers arrays (Periodical styleÑSubmitted for publication),Ó IEEE J. Quantum Electron., submitted for publication.
\bibitem{c7} C. J. Kaufman, Rocky Mountain Research Lab., Boulder, CO, private communication, May 1995.
\bibitem{c8} Y. Yorozu, M. Hirano, K. Oka, and Y. Tagawa, ÒElectron spectroscopy studies on magneto-optical media and plastic substrate interfaces(Translation Journals style),Ó IEEE Transl. J. Magn.Jpn., vol. 2, Aug. 1987, pp. 740Ð741 [Dig. 9th Annu. Conf. Magnetics Japan, 1982, p. 301].
\bibitem{c9} M. Young, The Techincal Writers Handbook.  Mill Valley, CA: University Science, 1989.
\bibitem{c10} J. U. Duncombe, ÒInfrared navigationÑPart I: An assessment of feasibility (Periodical style),Ó IEEE Trans. Electron Devices, vol. ED-11, pp. 34Ð39, Jan. 1959.
\bibitem{c11} S. Chen, B. Mulgrew, and P. M. Grant, ÒA clustering technique for digital communications channel equalization using radial basis function networks,Ó IEEE Trans. Neural Networks, vol. 4, pp. 570Ð578, July 1993.
\bibitem{c12} R. W. Lucky, ÒAutomatic equalization for digital communication,Ó Bell Syst. Tech. J., vol. 44, no. 4, pp. 547Ð588, Apr. 1965.
\bibitem{c13} S. P. Bingulac, ÒOn the compatibility of adaptive controllers (Published Conference Proceedings style),Ó in Proc. 4th Annu. Allerton Conf. Circuits and Systems Theory, New York, 1994, pp. 8Ð16.
\bibitem{c14} G. R. Faulhaber, ÒDesign of service systems with priority reservation,Ó in Conf. Rec. 1995 IEEE Int. Conf. Communications, pp. 3Ð8.
\bibitem{c15} W. D. Doyle, ÒMagnetization reversal in films with biaxial anisotropy,Ó in 1987 Proc. INTERMAG Conf., pp. 2.2-1Ð2.2-6.
\bibitem{c16} G. W. Juette and L. E. Zeffanella, ÒRadio noise currents n short sections on bundle conductors (Presented Conference Paper style),Ó presented at the IEEE Summer power Meeting, Dallas, TX, June 22Ð27, 1990, Paper 90 SM 690-0 PWRS.
\bibitem{c17} J. G. Kreifeldt, ÒAn analysis of surface-detected EMG as an amplitude-modulated noise,Ó presented at the 1989 Int. Conf. Medicine and Biological Engineering, Chicago, IL.
\bibitem{c18} J. Williams, ÒNarrow-band analyzer (Thesis or Dissertation style),Ó Ph.D. dissertation, Dept. Elect. Eng., Harvard Univ., Cambridge, MA, 1993. 
\bibitem{c19} N. Kawasaki, ÒParametric study of thermal and chemical nonequilibrium nozzle flow,Ó M.S. thesis, Dept. Electron. Eng., Osaka Univ., Osaka, Japan, 1993.
\bibitem{c20} J. P. Wilkinson, ÒNonlinear resonant circuit devices (Patent style),Ó U.S. Patent 3 624 12, July 16, 1990. 

\end{thebibliography}

\section{Below should be removed}

\subsection{Figures and Tables}

Positioning Figures and Tables: Place figures and tables at the top and bottom of columns. Avoid placing them in the middle of columns. Large figures and tables may span across both columns. Figure captions should be below the figures; table heads should appear above the tables. Insert figures and tables after they are cited in the text. Use the abbreviation ÒFig. 1Ó, even at the beginning of a sentence.

\begin{table}[h]
\caption{An Example of a Table}
\label{table_example}
\begin{center}
\begin{tabular}{|c||c|}
\hline
One & Two\\
\hline
Three & Four\\
\hline
\end{tabular}
\end{center}
\end{table}


   \begin{figure}[thpb]
      \centering
      \framebox{\parbox{3in}{We suggest that you use a text box to insert a graphic (which is ideally a 300 dpi TIFF or EPS file, with all fonts embedded) because, in an document, this method is somewhat more stable than directly inserting a picture.
}}
      %\includegraphics[scale=1.0]{figurefile}
      \caption{Inductance of oscillation winding on amorphous
       magnetic core versus DC bias magnetic field}
      \label{figurelabel}
   \end{figure}
   

Figure Labels: Use 8 point Times New Roman for Figure labels. Use words rather than symbols or abbreviations when writing Figure axis labels to avoid confusing the reader. As an example, write the quantity ÒMagnetizationÓ, or ÒMagnetization, MÓ, not just ÒMÓ. If including units in the label, present them within parentheses. Do not label axes only with units. In the example, write ÒMagnetization (A/m)Ó or ÒMagnetization {A[m(1)]}Ó, not just ÒA/mÓ. Do not label axes with a ratio of quantities and units. For example, write ÒTemperature (K)Ó, not ÒTemperature/K.Ó






\end{document}
Figure Labels: Use 8 point Times New Roman for Figure labels. Use words rather than symbols or abbreviations when writing Figure axis labels to avoid confusing the reader. As an example, write the quantity ÒMagnetizationÓ, or ÒMagnetization, MÓ, not just ÒMÓ. If including units in the label, present them within parentheses. Do not label axes only with units. In the example, write ÒMagnetization (A/m)Ó or ÒMagnetization {A[m(1)]}Ó, not just ÒA/mÓ. Do not label axes with a ratio of quantities and units. For example, write ÒTemperature (K)Ó, not ÒTemperature/K.Ó
% > \end{document}
%
% ^.
